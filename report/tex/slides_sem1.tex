%!TEX TS-program = xelatex

% Данный шаблон подготовлен для курса LaTeX в РАНХиГС
% на основе шаблона 
% Данила Фёдоровых (danil@fedorovykh.ru),
%  который использовал его в курсе 
% <<Документы и презентации в \LaTeX>> НИУ ВШЭ
% Исходная версия шаблона --- 
  % https://www.writelatex.com/coursera/latex/5.1
\documentclass[c, dvipsnames]{beamer}  % [t], [c], или [b] --- вертикальное 
%\documentclass[handout, dvipsnames, c]{beamer} % Раздаточный материал (на слайдах всё сразу)
\input{preamble}
\title[МСР в макроэкономике]{Оценка макроэкономических зависимостей с использованием методов снижения размерности в данных}
\subtitle{Отчёт по научно-исследовательской работе}


\author[Михаил Гареев]{Михаил Гареев \\ \smallskip \scriptsize ЭО-15-01 \\ \smallskip \scriptsize \href{mailto:mkhlgrv@gmail.com}{\nolinkurl{mkhlgrv@gmail.com} }}

\superviser{к.э.н. Полбин А.В.}

%\author[Имя автора]{Имя автора \\ \smallskip \scriptsize \href{mailto:author@ranepa.ru}{author@ranepa.ru} \\ \smallskip  \href{http://ranepa.ru}{http://ranepa.ru} }

\institute[РАНХиГС]{ \uppercase{
  Российская Академия Народного Хозяйства и  \\ Государственной Службы при Президенте Российской Федерации}}
\date{2018}


\titlegraphic{\includegraphics[scale=0.5]{logo/logo_ranepa.png}}
\titlegraphicii{\includegraphics[scale=0.5]{logo/logo_emit.png}}

\begin{document}

\frame[plain]{\titlepage}	% Титульный слайд


\begin{frame}[c]{Актуальность исследования} 
\begin{itemize}
\item  При оценке моделей из макроэкономики часто можно столкнуться с тем, что параметров относительно много, а наблюдений - мало. Иногда эту проблему решается использованием \alert{методов снижения размерности в данных}.
\end{itemize}
\end{frame}


% \begin{frame}[c]
% \frametitle{Анализ предметной области}
% { \small   % вместо small можно поставить scriptsize чтобы влезло
% 	\begin{table}[]
% 		\centering
% 		\resizebox{\textwidth}{!}{ 
%   			\begin{tabular}{|p{2.2cm}|p{1.8cm}|p{3.5cm}|p{7cm}|}
%   				\hline\rowcolor{backgr}
%   				\textcolor{white}{Авторы} & \textcolor{white}{Выборка, период}  & \textcolor{white}{Метод исследования}& \textcolor{white}{Результат} \\			
%   				\hline
%   				(Kuper, 2003)  & США, 998-2008  &  Коинтеграци и VECM & Получились значимые результаты с интересной интерпретацией.\\
%   				\hline
%   		\end{tabular} }
% 	\end{table}
% }
% \end{frame}


\begin{frame}[shrink=3]
	\frametitle{Цели и задачи}
	\begin{block}{Цель:}
	\begin{itemize}
		\item  Проверка некоторых гипотез макроэкономики при помощи методов
снижения размерности в данных и создание на их основе предсказательных моделей
для макроэкономических показателей.
	\end{itemize}
		
	\end{block}

	 	\begin{block}{Задачи:}
			\begin{enumerate}
	\item Обзор методов снижения размерности (LASSO, Post-LASSO, Dantzig Selector и др.).
\item Применение этих методов для оценки макроэкономических
зависимостей, анализ результатов, сравнение с другими методами оценивания и с
результатами, полученными ранее. 
\item Построение предсказательных моделей.
\item Создание процедуры мэтчинга стран на основе их макроэкономических показателей.
	 \end{enumerate}	
	\end{block}
\end{frame}



\section{Методы снижения размерности}
\subsection{Разреженная линейная модель с высокой размерностью в данных}

\begin{frame}
\frametitle{\insertsection} 
\framesubtitle{\insertsubsection}
Модель: 
  \begin{equation} 
\beta_0 + \varepsilon_i, \epsilon_i \sim N(0, \sigma^2), \beta_0 \in
\mathbb{R}^p, i = 1, \dots, n, 
\end{equation}
где:
  \begin{itemize}
\item $y_i$ --- это значения объясняемой
переменной, 
\item $x_i$ --- это значения $p$-размерной объясняющей переменной,
\item $\varepsilon_i$ --- значения независимых случайных ошибок в каждом наблюдении $i$, 
\end{itemize}
при этом возможно, что $p \geq n$, но только $s<n$ компонентов вектора$\beta_0$ не равны $0$.

\alert{Можно ли уменьшить размерность модели?}
\end{frame}

\subsection{Oracle Problem}
\begin{frame}[shrink=5]
\frametitle{\insertsection} 
\framesubtitle{\insertsubsection}
\begin{block}{Задача (Oracle Problem):}
\begin{equation}\label{op}
    \min_{\beta \in
\mathbb{R}^p} \mathbb{E}_n\left[ (y_i - {x_i}^{'} \beta)^2 \right] + \sigma^2
\frac{\left\lVert \beta \right\rVert_0}{n}, 
\end{equation} 
где $\left\lVert \beta \right\rVert_0$ --- это количество ненулевых компонентов в векторе $\beta$,  обобщение понятия нормы для степени $0$. 
\end{block}
\begin{block}{Гёльдерова норма для вектора $x$ степени $p$: }
\begin{equation}
    \left\lVert x \right\rVert_p = \sqrt[p]{\sum_i|x_i|^p},
\end{equation}
где обычно $p \geq 1$.

\end{block}

Решение \eqref{op} --- это  баланс между ошибкой регрессии и количеством ненулевых коэффициентов из вектора $\beta$. 

Методы снижения размерности оптимизируют эмпирические аналоги задачи \eqref{op}.
\end{frame}



\subsection{}



\begin{frame} 
\frametitle{\insertsection}  
\framesubtitle{\insertsubsection}
 	\begin{block}{AIC/ BIC}
  \begin{equation}
  \hat{\beta}  \in \arg \min_{\beta \in
\mathbb{R}^p} \sum_i=1^n \left[ (y_i - {x_i}^{'} \beta)^2 \right] +  \frac{\lambda}{n} \left\lVert \beta \right\rVert_0, 
\end{equation}
где $\lambda$ --- параметр штрафа.
 \end{block}

 \begin{block}{LASSO}
  \begin{equation}
  \hat{\beta}^{\text{LASSO}} \in \arg \min_{\beta \in
\mathbb{R}^p} \sum_i=1^n \left[ (y_i - {x_i}^{'} \beta)^2 \right] +  \frac{\lambda}{n} \left\lVert \beta \right\rVert_1,
\end{equation}
где $\lambda$ --- параметр штрафа, выбирается алгоритмически.
 \end{block}
 
\end{frame}

\begin{frame}
	\frametitle{\insertsection} 
\framesubtitle{\insertsubsection}

\begin{block}{Post-LASSO}
	\begin{enumerate}
		\item Использовать метода LASSO, найти $\hat{\beta}$.
		\item Применить МНК-регрессию, оценивая только неисключенные параметры $\beta$:
		\begin{equation}
		\tilde{\beta} \in \arg \min_{\beta \in
			\mathbb{R}^p} \sum_i=1^n \left[ (y_i - {x_i}^{'} \beta)^2 \right] +  \frac{\lambda}{n} \left\lVert \beta \right\rVert_1, \beta_j = 0| /hat{beta_j} = 0.
		\end{equation}
	\end{enumerate}
\end{block}


\begin{block}{Dantzig Selector}
	

\begin{eqnarray}
  \hat{\beta^{/text{DS}}} \min_{\beta \in
	\mathbb{R}^p} \left\lVert \beta \right\rVert_1\\
\text{s.t. } \left|x_i\beta - y_i\right| \leq \lambda \forall i = 1,\dots,n,
\end{eqnarray}

где $\lambda$ --- параметр штрафа, выбирается алгоритмически.

 \end{block}

\end{frame}

\section{Проверка гипотезы конвергенции с помощью методов снижения размерности}
\subsection{Однофакторная модель}
\begin{frame}
\frametitle{\insertsection} 
\framesubtitle{\insertsubsection}
    \begin{block}{Модель}
    \begin{equation}
g_i = \alpha +\beta ln(G_i) + \varepsilon_i, \epsilon_i \sim N(0, \sigma^2),
\end{equation}
где:
\begin{itemize}
    \item $g_i$ --- средний за 1980--1984 темп роста реального ВВП на душу населения, 
    \item $G_i$ --- логарифим ВВП на душу населения в 1980 г. (в долларах) для страны $i, i =1, \dots, 245$.
\end{itemize}
    \end{block}
    Данные:
    
\end{frame}


\section{Проверка гипотезы конвергенции с помощью методов снижения размерности.}
\subsection{Использование LASSO}
\begin{frame}
\frametitle{\insertsection} 
\framesubtitle{\insertsubsection}
    \begin{block}{Модель}
    \begin{equation}
g_i = \alpha +\beta ln(G_i) + \varepsilon_i, \epsilon_i \sim N(0, \sigma^2),
\end{equation}
где:
\begin{itemize}
    \item $g_i$ --- средний за 1980--1984 темп роста реального ВВП на душу населения, 
    \item $G_i$ --- логарифим ВВП на душу населения в 1980 г. (в долларах) для страны $i, i =1, \dots, 245$.
\end{itemize}
    \end{block}
    
    
\end{frame}
\subsection{Данные}
\begin{frame}
\frametitle{\insertsection} 
\framesubtitle{\insertsubsection}
    \includegraphics[width=\textwidth]{plot/gdp2growth_BL.png}
\end{frame}

\subsection{Сравнение результатов}
\begin{frame}
\frametitle{\insertsection} 
\framesubtitle{\insertsubsection}
{\scriptsize
\begin{table}[!htbp] \centering 
  \caption{Результаты регрессий} 
  \label{} 
\begin{tabular}{@{\extracolsep{5pt}}lcc} 
\\[-1.8ex]\hline 
\hline \\[-1.8ex] 
\\[-1.8ex] & \multicolumn{2}{c}{g} \\ 
\\[-1.8ex] & (1) & (2) \\ 
\hline \\[-1.8ex] 
 G & 0.001 ($-$0.005, 0.007) & $-$0.0112 ($-$0.022, 0.001) \\ 
  C & $-$0.010 ($-$0.055, 0.034) & $-$0.03 ($-$0.032, 0.041) \\ 
 \hline \\[-1.8ex] 
Наблюдений & 120 & 120 \\ 
R$^{2}$ & 0.001 & 0.001 \\ 
Adjusted R$^{2}$ & $-$0.007 & $-$0.007 \\ 
lambda & & 2.7870\\
\hline 
\hline \\[-1.8ex] 
\textit{Примечание:}  & \multicolumn{2}{r}{$^{*}$p$<$0.1; $^{**}$p$<$0.05; $^{***}$p$<$0.01} \\ 
\end{tabular} 
\end{table}
}


\end{frame}


\section{Краткий вывод и планы}
\begin{frame}
\frametitle{\insertsection} 
\begin{itemize}
    \item  Методы снижения размерности (LASSO, Post-LASSO и др.) потенциально представляют собой мощный инструмент для нахождения и проверки макроэкономических зависимостей.
    \item Планы на ближайшее время:
    \begin{itemize}
        \item  Сделать осмысленные LASSO, Post-Lasso регрессии для проверки гипотезы конвергенции на основе современных данных Всемирного банка.
        \item Проверить другие макроэкономические гипотезы с помощью методов снижения размерности.
    \end{itemize}
\end{itemize}
   
    
\end{frame}



\begin{frame}[c, plain]
\begin{center}

{\LARGE Спасибо за внимание}

\bigskip

{\Large \inserttitle}

\bigskip

{\insertauthor} 

\bigskip

\bigskip\bigskip

{\large \insertdate}
\end{center}
\end{frame}



\begin{frame}[c, plain]
  \frametitle{Источники}    
  \begin{thebibliography}{10}    
  \beamertemplatearticlebibitems
  \bibitem{ch1}
    Belloni, Alexandre and Chernozhukov, Victor
    \newblock High dimensional sparse econometric models: An introduction.
    \newblock Springer, 2011
   \bibitem{ch2}
   Belloni, Alexandre, Victor Chernozhukov, and Christian Hansen. 
   \newblock Lasso methods for gaussian instrumental variables models
   \newblock 2011
    \bibitem{bl}
    Barro, Robert J.  and Lee, Jong-Wha 
    \newblock Data Set for a Panel of 138 Countries
    \newblock 1994
    \bibitem{g}
     Candes, Emmanuel, and Terence Tao. 
    \newblock The Dantzig selector: Statistical estimation when p is much larger than n.
    \newblock The Annals of Statistics 35.6 (2007): 2313-2351.
    \bibitem{f}
     Akaike, Hirotugu.
    \newblock A new look at the statistical model identification.
    \newblock IEEE transactions on automatic control 19.6 (1974): 716-723.
  \end{thebibliography}
\end{frame}




\end{document}