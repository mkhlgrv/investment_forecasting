%!TEX TS-program = xelatex

% Данный шаблон подготовлен для курса LaTeX в РАНХиГС
% на основе шаблона 
% Данила Фёдоровых (danil@fedorovykh.ru),
%  который использовал его в курсе 
% <<Документы и презентации в \LaTeX>> НИУ ВШЭ
% Исходная версия шаблона --- 
  % https://www.writelatex.com/coursera/latex/5.1
\documentclass[c, dvipsnames]{beamer}  % [t], [c], или [b] --- вертикальное 
%\documentclass[handout, dvipsnames, c]{beamer} % Раздаточный материал (на слайдах всё сразу)
\input{preamble}
\title[МСР в макроэкономике]{Методы снижения размерности в данных в макроэкономике}
\subtitle{Отчёт по научно-исследовательской работе}


\author[Михаил Гареев]{Михаил Гареев \\ \smallskip \scriptsize ЭО-15-01 \\ \smallskip \scriptsize \href{mailto:mkhlgrv@gmail.com}{\nolinkurl{mkhlgrv@gmail.com} }}

\superviser{к.э.н. Полбин А.В.}

%\author[Имя автора]{Имя автора \\ \smallskip \scriptsize \href{mailto:author@ranepa.ru}{author@ranepa.ru} \\ \smallskip  \href{http://ranepa.ru}{http://ranepa.ru} }

\institute[РАНХиГС]{ \uppercase{
  Российская Академия Народного Хозяйства и  \\ Государственной Службы при Президенте Российской Федерации}}
\date{2019}


\titlegraphic{\includegraphics[scale=0.5]{logo/logo_ranepa.png}}
\titlegraphicii{\includegraphics[scale=0.5]{logo/logo_emit.png}}

\begin{document}

\frame[plain]{\titlepage}	% Титульный слайд


\begin{frame}[c]{Актуальность исследования} 
\begin{itemize}
\item  При оценке моделей из макроэкономики часто можно столкнуться с тем, что параметров относительно много, а наблюдений - мало. Иногда эта проблема решается использованием \alert{методов снижения размерности в данных}.
\end{itemize}
\end{frame}


% \begin{frame}[c]
% \frametitle{Анализ предметной области}
% { \small   % вместо small можно поставить scriptsize чтобы влезло
% 	\begin{table}[]
% 		\centering
% 		\resizebox{\textwidth}{!}{ 
%   			\begin{tabular}{|p{2.2cm}|p{1.8cm}|p{3.5cm}|p{7cm}|}
%   				\hline\rowcolor{backgr}
%   				\textcolor{white}{Авторы} & \textcolor{white}{Выборка, период}  & \textcolor{white}{Метод исследования}& \textcolor{white}{Результат} \\			
%   				\hline
%   				(Kuper, 2003)  & США, 998-2008  &  Коинтеграци и VECM & Получились значимые результаты с интересной интерпретацией.\\
%   				\hline
%   		\end{tabular} }
% 	\end{table}
% }
% \end{frame}


\begin{frame}[shrink=3]
	\frametitle{Цели и задачи}
	\begin{block}{Цель:}
	\begin{itemize}
		\item Проверка целесообразности использования методов снижения размерности в данных в прогнозировании российских макроэкономических показателей.
	\end{itemize}
		
	\end{block}

	 	\begin{block}{Задачи:}
			\begin{enumerate}
	\item Обзор методов снижения размерности (LASSO, Post-LASSO, Ridge, Elastic Net, PCA, Random Forest, Spike-and-Slab variable selection).
    \item Применение этих методов для прогнозирования макроэкономических
рядов в России (безработицы), анализ результатов, сравнение с традиционными методами оценивания временных рядов.
	 \end{enumerate}	
	\end{block}
\end{frame}



\section{Методы снижения размерности}
\subsection{Разреженная линейная модель с высокой размерностью в данных}

\begin{frame}[shrink=5]
\frametitle{\insertsection} 
\framesubtitle{\insertsubsection}
Модель: 
  \begin{equation} 
\beta_0 + \varepsilon_i, \epsilon_i \sim N(0, \sigma^2), \beta_0 \in
\mathbb{R}^p, i = 1, \dots, n, 
\end{equation}
где:
  \begin{itemize}
\item $y_i$ --- это значения объясняемой
переменной, 
\item $x_i$ --- это значения $p$-размерной объясняющей переменной,
\item $\varepsilon_i$ --- значения независимых случайных ошибок в каждом наблюдении $i$, 
\end{itemize}
при этом возможно, что $p \geq n$, но только $s<n$ компонентов вектора $\beta_0$ не равны $0$.

\item \alert{Можно ли уменьшить размерность модели?}
\end{frame}

\subsection{Oracle Problem}
\begin{frame}[shrink=5]
\frametitle{\insertsection} 
\framesubtitle{\insertsubsection}
\begin{block}{Задача (Oracle Problem):}
\begin{equation}\label{op}
    \min_{\beta \in
\mathbb{R}^p} \mathbb{E}_n\left[ (y_i - {x_i}^{'} \beta)^2 \right] + \sigma^2
\frac{\left\lVert \beta \right\rVert_0}{n}, 
\end{equation} 
где $\left\lVert \beta \right\rVert_0$ --- это количество ненулевых компонентов в векторе $\beta$,  обобщение понятия нормы для степени $0$. 
\end{block}
\begin{block}{Гёльдерова норма для вектора $x$ степени $p$: }
\begin{equation}
    \left\lVert x \right\rVert_p = \sqrt[p]{\sum_i|x_i|^p},
\end{equation}
где обычно $p \geq 1$.

\end{block}

Решение \eqref{op} --- это  баланс между ошибкой регрессии и количеством ненулевых коэффициентов из вектора $\beta$. 

Методы снижения размерности оптимизируют эмпирические аналоги задачи \eqref{op}.
\end{frame}



\subsection{Регуляризация}



\begin{frame}
\frametitle{\insertsection} 
\framesubtitle{\insertsubsection}
 \begin{block}{AIC/ BIC}
  \begin{equation}
  \hat{\beta}  \in \arg \min_{\beta \in
\mathbb{R}^p} \sum_{i=1}^n \left[ (y_i - {x_i}^{'} \beta)^2 \right] +  \frac{\lambda}{n} \left\lVert \beta \right\rVert_0, 
\end{equation}
где $\lambda$ --- параметр штрафа.
 \end{block}

 \begin{block}{LASSO}
  \begin{equation}
  \hat{\beta}^{\text{LASSO}} \in \arg \min_{\beta \in
\mathbb{R}^p} \sum_{i=1}^n \left[ (y_i - {x_i}^{'} \beta)^2 \right] +  \frac{\lambda}{n} \left\lVert \beta \right\rVert_1,
\end{equation}
где $\lambda$ --- параметр штрафа.
 \end{block}
 
  \end{frame}
 \begin{frame}
 \frametitle{\insertsection} 
\framesubtitle{\insertsubsection}

 \begin{block}{Ridge Regression}
  \begin{equation}
  \hat{\beta}^{\text{Ridge}} \in \arg \min_{\beta \in
\mathbb{R}^p} \sum_{i=1}^n \left[ (y_i - {x_i}^{'} \beta)^2 \right] +  \frac{\lambda}{n} \left\lVert \beta \right\rVert_2,
\end{equation}
где $\lambda$ --- параметр штрафа.
 \end{block}

 \begin{block}{Elastic Net Regression}
  \begin{equation}
  \hat{\beta}^{\text{EN}} \in \arg \min_{\beta \in
\mathbb{R}^p} \sum_{i=1}^n \left[ (y_i - {x_i}^{'} \beta)^2 \right] +  \frac{\lambda}{n}
\left( \frac{1 - \alpha}{2} \left\lVert \beta \right\rVert_1 + 
\alpha \left\lVert \beta \right\rVert_2 \right),
\end{equation}
где $\lambda$ --- параметр штрафа,
$\alpha$ --- параметр регуляризации, равен $1$ для Ridge и
$2$ для LASSO.
\end{block}
 
 
\end{frame}
\begin{frame}
\frametitle{\insertsection} 
\framesubtitle{\insertsubsection}
\begin{block}{Post-LASSO}
\begin{enumerate}
    \item Использовать метода LASSO, найти $\hat{\beta}^{\text{LASSO}}$.
    \item Применить МНК-регрессию, оценивая только неисключенные элементы $\hat{\beta}^{\text{LASSO}}$:
    \begin{equation}
  \hat{\beta}^{\text{Post-LASSO}} \in \arg \min_{\beta \in
\mathbb{R}^p} \sum_{i=1}^n \left[ (y_i - {x_i}^{'} \beta)^2 \right] \text{,  где }  \beta_j = 0 \text{, если } \hat{\beta_j} = 0.
\end{equation}
\end{enumerate}
\end{block}
 
 


\end{frame}



\subsection{Ансамблевые методы}
\begin{frame}
\frametitle{\insertsection} 
\framesubtitle{\insertsubsection}
\begin{block}{Random Forest}
    Двухэтапное получение оценок:
    \begin{enumerate}
        \item На разных подвыборках данных строится множество решающих деревьев,
        \item в качестве предсказанного значения $\hat{y_i}$ выбираются усреднённые значения показаний по всем деревьям.
    \end{enumerate}
\end{block}


\end{frame}

\subsection{Байесовские методы}
\begin{frame}
\frametitle{\insertsection} 
\framesubtitle{\insertsubsection}

\begin{block}{Регрессия пик-плато (Spike-and-slab)}

    \begin{itemize}
        \item $\beta_j|\tau_j, r^2_j \sim N(0,\tau_j\cdot r^2_j )$
        \item \begin{equation}
             \tau_j = 
 \begin{cases}
   0 &\text{se $\omega\in A$}\\
   1 &\text{se $\omega \in A^c$}
 \end{cases}
        \end{equation}
        \item $r^2_j \sim \text{Exp}(\lambda)$ 
    \end{itemize}
    
    
\end{block}


\end{frame}


\section{Прогнозирование безработицы}
\subsection{Описание данных}
\begin{frame}
\frametitle{\insertsection} 
\framesubtitle{\insertsubsection}
\begin{enumerate}
    \item Прогнозируемая переменная: уровень безработицы в России (ноябрь 2001 -- декарь 2017),
    \item Объясняющие переменные: 83 ряда данных, отражающие различные макроэкономические показатели в России, уровень деловой активности и др. (январь 2001 -- декарь 2016).
\end{enumerate}
    Обучение моделей ведется на десятилетнем движущемся окне, проверка качества моделей ведется на однолетнем окне для изменения безработицы в период от 1 до 24 месяцев.
    
    Все ряды были очищены от сезонных и календарных эффектов и приведены к стационарному виду.
\end{frame}

\begin{frame}
\frametitle{\insertsection} 
\framesubtitle{\insertsubsection}
\begin{figure*}
\caption{Безработица в России}
\includegraphics[width=\linewidth]{unemp_level.pdf}
\end{figure*}
\end{frame} 

\subsection{Метод главных компонент}
\begin{frame}
\frametitle{\insertsection} 
\framesubtitle{\insertsubsection}
PCA: последовательная минимизация суммы квадратов отклонений старых значений от новых или замена матрицы $X_{n \times k}$ на матрицу $\hat{{n \times k}}$ ранга $p < k$, так, чтобы:

\begin{equation}
    \min \sum_{j = 1}^k \sum_{i = 1}^{n}(x_{ij} - \hat{x_{ij}})^2
\end{equation}

\end{frame}


\begin{frame}
\frametitle{\insertsection} 
\framesubtitle{\insertsubsection}
\begin{figure*}
\caption{Экономика России в двумерном пространстве}
\includegraphics[width=\linewidth]{pca.pdf}
\end{figure*}
\end{frame}  



\subsection{Базовый бенчмарк}
\begin{frame}
\frametitle{\insertsection} 
\framesubtitle{\insertsubsection}
    \begin{block}{Модель ARMA(p,q)}
Для сравнения качества используется модель ARMA(p,q), где p и q выбираются при помощи AIC.
    \end{block}
 \end{frame}   
 

 \subsection{Метрика качества моделей}
\begin{frame}
\frametitle{\insertsection} 
\framesubtitle{\insertsubsection}
    \begin{block}{RMSE}
Для сравнения качества используется метрика RMSE (Root-mean-square error):
\begin{equation}
   \text{RMSE} = \sqrt{ \frac{\sum_{t = 1}^{T} (\hat{y_t} - y_t)}{T}} 
\end{equation}

    \end{block}
 \end{frame}  

\subsection{Результаты}


\begin{frame}
\frametitle{\insertsection} 
\framesubtitle{\insertsubsection}
\begin{figure*}
\caption{Базовый прогноз}
\includegraphics[width=\linewidth]{arma.pdf}
\end{figure*}
\end{frame}


\begin{frame}
\frametitle{\insertsection} 
\framesubtitle{\insertsubsection}
\begin{figure*}
\caption{Модели с регуляризацией}
\includegraphics[width=\linewidth]{reg_score.pdf}
\end{figure*}
\end{frame}


\begin{frame}
\frametitle{\insertsection} 
\framesubtitle{\insertsubsection}
\begin{figure*}
\caption{Модели с регуляризацией и трансформацией данных с помощью главных компонент}
\includegraphics[width=\linewidth]{regpc_score.pdf}
\end{figure*}
\end{frame}


\begin{frame}
\frametitle{\insertsection} 
\framesubtitle{\insertsubsection}
\begin{figure*}
\caption{Остальные модели}
\includegraphics[width=\linewidth]{rfss_score.pdf}
\end{figure*}
\end{frame}


\begin{frame}
\frametitle{\insertsection} 
\framesubtitle{\insertsubsection}
\begin{figure*}
\caption{Сравнения предсказаний для некоторых моделей (1--9 мес)}
\includegraphics[width=\linewidth]{bl1.pdf}
\end{figure*}
\end{frame}


\begin{frame}
\frametitle{\insertsection} 
\framesubtitle{\insertsubsection}
\begin{figure*}
\caption{Сравнения предсказаний для некоторых моделей (10--18 мес)}
\includegraphics[width=\linewidth]{bl2.pdf}
\end{figure*}
\end{frame}
 
 \frametitle{\insertsection} 
\framesubtitle{\insertsubsection}
 \begin{frame}
\begin{table}[ht]
\small
\centering
\begin{tabular}{lrrrrrr}
  \hline
Модель & 1 & 2 & 3 & 4 & 5 & 6 \\ 
  \hline
Elastic Net & 0.97 &  & 0.85 &  &  &  \\ 
  LASSO & 0.97 & 0.85 & 0.85 & 0.86 & 0.84 & 0.83 \\ 
  LASSO with lag & 0.89 & 0.84 & 0.83 & 0.92 & 0.94 & 1.02 \\ 
 LASSO with PC & 0.97 & 0.82 & 0.80 & 0.83 & 0.86 & 0.96 \\ 
 LASSO with PC and lag &0.89 & 0.93 & 1.05 & 1.10 & 1.16 & 1.30 \\ 
  Post-LASSO & 1.03 & 0.91 & 1.03 & 0.97 & 1.02 & 1.01 \\ 
 Post-LASSO with lag & 0.95 & 1.02 & 1.09 & 1.15 & 1.32 & 1.59 \\ 
  Post-LASSO with PC & 1.01 & 0.98 & 0.95 & 1.06 & 1.34 & 1.56 \\ 
  Post-LASSO with PC  and lag & 1.11 & 0.98 & 1.43 & 1.25 & 1.34 & 1.41 \\ 
  Random Forest & 1.00 & 0.87 & 0.91 & 0.95 & 0.99 & 1.16 \\ 
  Ridge & 1.35 & 1.16 & 1.05 & 1.16 & 1.38 & 1.37 \\ 
  Ridge with PC & 1.43 & 1.31 & 1.27 & 1.18 & 1.17 & 1.17 \\ 
   Spike-and-Slab & 1.02 & 0.91 & 1.03 & 1.20 & 1.27 & 1.41 \\
   \hline
\end{tabular}
\end{table}
\end{frame}

 \begin{frame}
% latex table generated in R 3.5.2 by xtable 1.8-3 package
% Thu Mar 28 16:06:58 2019
\begin{table}[ht]
\small
\centering
\begin{tabular}{lrrrrrr}
  \hline
  
Модель & 7 & 8 & 9 & 10 & 11 & 12 \\ 
  \hline
Elastic Net &  &  &  &  &  & 1.88 \\ 
  LASSO & 0.90 & 0.99 & 1.13 & 1.43 & 1.53 & 1.88 \\ 
  LASSO with lag & 1.12 & 1.15 & 1.20 & 1.40 & 1.54 & 1.88 \\ 
 LASSO with PC & 1.05 & 1.10 & 1.12 & 1.32 & 1.33 & 1.45 \\ 
 LASSO with PC and lag &1.50 & 1.58 & 1.55 & 1.78 & 1.85 & 2.04 \\ 
  Post-LASSO & 1.47 & 1.54 & 1.56 & 1.70 & 1.82 & 2.11 \\ 
 Post-LASSO with lag & 1.61 & 1.69 & 1.84 & 2.06 & 2.32 & 3.16 \\ 
  Post-LASSO with PC & 1.61 & 1.53 & 1.52 & 1.82 & 1.85 & 2.05 \\ 
  Post-LASSO with PC  and lag & 1.94 & 1.94 & 1.91 & 2.15 & 2.07 & 2.45 \\ 
  Random Forest & 1.06 & 1.05 & 1.02 & 1.16 & 1.34 & 1.26 \\ 
  Ridge & 1.44 & 1.48 & 1.74 & 2.22 & 2.56 & 3.04 \\ 
  Ridge with PC & 1.33 & 1.24 & 1.36 & 1.64 & 1.97 & 2.30 \\ 
  Spike-and-Slab & 1.24 & 1.32 & 1.78 & 2.20 & 2.62 & 3.18 \\
   \hline
\end{tabular}
\end{table}
\end{frame}

 \begin{frame}
% latex table generated in R 3.5.2 by xtable 1.8-3 package
% Thu Mar 28 16:07:55 2019
\begin{table}[ht]
\centering
\small
\begin{tabular}{lrrrrrr}
  \hline
Модель & 13 & 14 & 15 & 16 & 17 & 18 \\ 
  \hline
Elastic Net &  &  &  &  &  &  \\ 
  LASSO & 1.88 & 1.83 & 1.65 & 1.41 & 1.47 & 1.59 \\ 
  LASSO with lag & 1.94 & 2.15 & 2.02 & 1.84 & 1.56 & 1.45 \\ 
 LASSO with PC & 1.53 & 1.59 & 1.50 & 1.44 & 1.47 & 1.61 \\ 
 LASSO with PC and lag &2.02 & 2.12 & 1.95 & 1.75 & 1.58 & 1.62 \\ 
  Post-LASSO & 2.34 & 2.48 & 2.70 & 2.47 & 2.60 & 2.79 \\ 
 Post-LASSO with lag & 3.54 & 3.60 & 3.32 & 3.21 & 2.88 & 2.85 \\ 
  Post-LASSO with PC & 2.41 & 2.48 & 2.32 & 2.22 & 2.09 & 2.25 \\ 
  Post-LASSO with PC  and lag & 2.08 & 2.51 & 2.29 & 1.87 & 1.66 & 1.91 \\ 
  Random Forest & 1.22 & 1.18 & 1.04 & 0.91 & 0.81 & 0.82 \\ 
  Ridge & 3.27 & 3.50 & 3.37 & 3.10 & 3.00 & 2.94 \\ 
  Ridge with PC & 2.58 & 2.70 & 2.64 & 2.36 & 2.28 & 2.25 \\ 
  Spike-and-Slab & 3.47 & 3.51 & 3.49 & 3.16 & 3.18 & 3.07 \\  
   \hline
\end{tabular}
\end{table}
\end{frame}




\section{Краткий вывод и планы}
\begin{frame}
\frametitle{\insertsection} 
\begin{itemize}
    \item  Методы снижения размерности (LASSO, Post-LASSO, Ridge, Elastic Net, Random Forest) потенциально представляют собой мощный инструмент для нахождения и проверки макроэкономических зависимостей.
    \item Из использованных методов лучшие результаты при прогнозировании инфляции в России показывают модели LASSO и Random Forest. На разных горизонтах планирования  (кроме диапазона с 9 до 15 месяцев) хотя бы одна из них показывала лучшие результаты, чем модель-бенчмарк (ARMA).
\end{itemize}
   
    
\end{frame}



\begin{frame}[c, plain]
\begin{center}

{\LARGE Спасибо за внимание}

\bigskip

{\Large \inserttitle}

\bigskip

{\insertauthor} 

\bigskip

\bigskip\bigskip

{\large \insertdate}
\end{center}
\end{frame}



\begin{frame}[c, plain]
  \frametitle{Источники}    
  \begin{thebibliography}{8}    
  \beamertemplatearticlebibitems
  {\small
  \bibitem{ch1}
    Belloni, Alexandre and Chernozhukov, Victor
    \newblock High dimensional sparse econometric Модельs: An introduction.
    \newblock Springer, 2011
   \bibitem{ch2}
   Belloni, Alexandre, Victor Chernozhukov, and Christian Hansen. 
   \newblock Lasso methods for gaussian instrumental variables Модельs
   \newblock 2011
    \bibitem{bl}
    Barro, Robert J.  and Lee, Jong-Wha 
    \newblock Data Set for a Panel of 138 Countries
    \newblock 1994
    \bibitem{g}
     Candes, Emmanuel, and Terence Tao. 
    \newblock The Dantzig selector: Statistical estimation when p is much larger than n.
    \newblock The Annals of Statistics 35.6 (2007): 2313-2351.
    \bibitem{f}
     Akaike, Hirotugu.
    \newblock A new look at the statistical Модель identification.
    \newblock IEEE transactions on automatic control 19.6 (1974): 716-723.
    \bibitem{s}
    Единый архив экономических и социологических данных, статистические ряды
    \newblock http://sophist.hse.ru/hse/nindex.shtml
    \newblock 
    }
  \end{thebibliography}
\end{frame}




\end{document}